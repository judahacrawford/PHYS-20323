%%%%%%%%%%%%%%%%%%%%%%%%%%%%%%%%%%%%%%%%%%%%%%%%%%%%%%%%%%%%
%%%%%%%%%%%%%%%%%%%%%%%%%%%%%%%%%%%%%%%%%%%%%%%%%%%%%%%%%%%%
%%%%%%%%%%%%%%%%%%%%%%%%%%%%%%%%%%%%%%%%%%%%%%%%%%%%%%%%%%%%
%%%%%%%%%%%%%%%%%%%%%%%%%%%%%%%%%%%%%%%%%%%%%%%%%%%%%%%%%%%%
%%%%%%%%%%%%%%%%%%%%%%%%%%%%%%%%%%%%%%%%%%%%%%%%%%%%%%%%%%%%
\documentclass[12pt]{article}
\usepackage{epsfig}
\usepackage{times}
\usepackage{amsmath}
\usepackage{xcolor}
\usepackage{fancyhdr}
\renewcommand{\topfraction}{1.0}
\renewcommand{\bottomfraction}{1.0}
\renewcommand{\textfraction}{0.0}
\setlength {\textwidth}{6.6in}
\hoffset=-1.0in
\oddsidemargin=1.00in
\marginparsep=0.0in
\marginparwidth=0.0in                                                                               
\setlength {\textheight}{9.0in}
\voffset=-1.00in
\topmargin=1.0in
\headheight=0.0in
\headsep=0.00in
\footskip=0.50in                                         
\setcounter{page}{1}
\fancyfoot{}
\pagestyle{fancy}
\renewcommand{\headrulewidth}{0pt}
\fancyfoot[R]{31}
\fancyfoot[L]{Latex Example}
\begin{document}
\def\pos{\medskip\quad}
\def\subpos{\smallskip \qquad}
\newfont{\nice}{cmr12 scaled 1250}
\newfont{\name}{cmr12 scaled 1080}
\newfont{\swell}{cmbx12 scaled 800}

%%%%%%%%%%%%%%%%%%%%%%%%%%%%%%%%%%%%%%%%%%%%%%%%%%%%%%%%%%%%
%     DO NOT CHANGE ANYTHING ABOVE THIS LINE
%%%%%%%%%%%%%%%%%%%%%%%%%%%%%%%%%%%%%%%%%%%%%%%%%%%%%%%%%%%%
%     DO NOT CHANGE ANYTHING ABOVE THIS LINE
%%%%%%%%%%%%%%%%%%%%%%%%%%%%%%%%%%%%%%%%%%%%%%%%%%%%%%%%%%%%
%     DO NOT CHANGE ANYTHING ABOVE THIS LINE
%%%%%%%%%%%%%%%%%%%%%%%%%%%%%%%%%%%%%%%%%%%%%%%%%%%%%%%%%%%%

\begin{center}
{\LARGE
PHYSICS  20323/60323: Fall 2023 - LaTeX Example
}\\

%%%%%%%%%%%%%%%%%%%%%%%%%%%%%%%%%%%%%%%%%%%%%%%%%%%%%%%%%%%%
\end{center}
%%%%%%%%%%%%%%%%%%%%%%%%%%%%%%%%%%%%%%%%%%%%%%%%%%%%%%%%%%%%
% Section Heading
%%%%%%%%%%%%%%%%%%%%%%%%%%%%%%%%%%%%%%%%%%%%%%%%%%%%%%%%%%%%

\begin{enumerate}
\item {\bf The following questions refer to stars in the Table below.} \\
{Note: There may be multiple answers.} \\

%%%%%%%%%%%%%%%%%%%%%%%%%%%%%%%%%%%%%%%%%%%%%%%%%%%%%%%%%%%%
\begin{tabular}{|l|c|c|c|c|c|}
      \hline
      Name 
     & Mass
     & Luminosity
     & Lifetime
     & Temperature
     & Radius\\ 
     \hline
     $\eta$ Car.
     & 60. M{\footnotesize $\odot$}
     & $10^6$ L{\footnotesize $\odot$}
     & $8.0 \times 10^5$ years &&\\
     \hline
     $\epsilon$ Eri.
     & 6.0 M{\footnotesize $\odot$}
     & $10^3$ L{\footnotesize $\odot$} &
     & 20,000 K & \\
     \hline
     $\delta$ Scu.
     &2.0 M{\footnotesize $\odot$} &
     & $5.0 \times 10^8$ years &
     & 2 R{\footnotesize $\odot$}\\
     \hline
     $\beta$ Cyg.
     &1.3 M{\footnotesize $\odot$}
     &3.5 L{\footnotesize $\odot$}&&&\\
     \hline
     $\alpha$ Cen.
     &1.0 M{\footnotesize $\odot$}&&&
     & 1 R{\footnotesize $\odot$}\\
     \hline
     $\gamma$ Del.
     &0.7 M{\footnotesize $\odot$}&
     & $4.5 \times 10^10$ years 
     & 5000 K&\\
     \hline
     
\end{tabular}
\begin{enumerate}
\item (4 points) Which of these stars will produce a planetary nebula.
\\
\item (4 points) Elements heavier than Carbon will be produced in which stars.
\end{enumerate}
\item An electron is found to be in the spin state (in the z-basis): 
$\chi$ = A $\begin{pmatrix} 3 \textit{i} \\4 \end{pmatrix}$

\begin{enumerate}
    \item (5 points) Determine the possible values of A such that the state is normalized.
    \\
    \item (5 points) Find the expectation values of the operators {\color{red}$S_{x}$}, {\color{purple}$S_{y}$}, {\color{orange}$S_{z}$} and $\vec{S}^2$.\\
    
    

\end{enumerate}
The matrix representations in the z-basis for the components of electron spin operators are given by:\\

{\color{red}\bf{$S_{x}$} = $\frac{\hbar}{2}\begin{pmatrix} 0&1\\1&0\end{pmatrix}$}
{\color{purple};\hspace{1cm}\bf{$S_{y}$} = $\frac{\hbar}{2}\begin{pmatrix} 0&-\textit{i}\\\textit{i}&0\end{pmatrix}$}
{\color{orange};\hspace{1cm}\bf{$S_{z}$} = $\frac{\hbar}{2}\begin{pmatrix} 1&0\\0&-1\end{pmatrix}$}


\item The average electrostatic field in the earth’s atmosphere in fair weather is approximately given:


\begin{equation}
\vec{E} = E_{0}(Ae^{-\alpha z}+Be^{-\beta z})\hat{z}, 
\end{equation}

where A, B, $\alpha$, $\beta$ are positive constants and z is the height above the (locally flat) earth surface.
\begin{enumerate}
    \item (5 points) Find the average charge density in the atmosphere as a function of height\\
    \item (5 points) Find the electric potential as a function height above the earth.
\end{enumerate}

\end{enumerate}


\end{document}